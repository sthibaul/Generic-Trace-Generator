\section{Using the GTG library}
The GTG library provides a generic interface to writedown traces. Users are 
advised to use the generic calls (in the GTGBasic file) instead of using the
specifics one (OTF\_*, PAJE\_*). There are some steps that must be followed
to use the library :
\begin{itemize}
\item First, one sould choose the kind of trace to create
\item Then, the init function has to be called
\item \textcolor{red}{Warning, Changing the kind of trace to generate after
the initialisation is to result in an undefined behavior (mostly segfault)}
\item Call all the trace writting functions
\item You must end with the finalize call to clean the internal memory
\end{itemize}

Another usefull interface is the color one, to customize the states 
information, one can set color to the states. The color API defines
colors, but the user can create his own color using a RGB format.

Some examples of using the API are provided within the \textit{exm} directory
in the package.
